\documentclass[a4paper, 14pt, russian]{article}
\usepackage[a4paper]{geometry}
\usepackage{mathtext}
\usepackage{lipsum}
\usepackage{extsizes}
\usepackage{cmap}
\usepackage{caption}
\usepackage[T2A]{fontenc}
\usepackage[utf8]{inputenc}
\usepackage[russian]{babel}
\usepackage{physics}
\usepackage{hyperref}
\usepackage{indentfirst}
\usepackage{fancyhdr}
\usepackage{enumerate}
\usepackage{amssymb, amsmath}
\usepackage{tikz}
\usepackage{pgfplots}
\usepackage{mdwlist}
\usepackage{apacite}
\geometry{left=3cm,right=1.5cm,top=2cm,bottom=2cm}
\usepackage[unicode]{hyperref}  %создаёт гиперссылки на список литературы в pdf-файле
\linespread{1.3}                % полтора интервала. Если 1.6, то два интервала
\begin{document}
\begin{titlepage}
\begin{center}
\textbf{МИНИСТЕРСТВО НАУКИ И ВЫСШЕГО ОБРАЗОВАНИЯ РОССИЙСКОЙ ФЕДЕРАЦИИ}

\vspace{0.3cm}

\textbf{Национальный исследовательский университет}

\textbf{Высшая школа экономики}

\vspace{0.3cm}

Факультет информатики, математики и компьютерных наук
\end{center}


\begin{center}
\Large{\bf{Вероятностная модель транспортных потоков на автомагистрали}}
\end{center}

\vspace{0.3cm}

\begin{flushright}
Курсовая работа

студента 1 курса магистратуры

специальности ''Интеллектуальный анализ данных'',

Шилова Андрея Сергеевича

% \vspace{6\mytextsize}
\vspace{0.5 cm}
\underline{Научный руководитель:}

Кандидат физико-математических наук

Федоткин Андрей Михайлович

\end{flushright}

\vspace{9cm}

\begin{center}
Нижний Новгород

2020 г.
\end{center}

\end{titlepage}

\tableofcontents                % Автоматическое создание оглавления по названиям разделов, подразделов и т.п.

\newpage

\section{Введение}

С развитием транспортной индустрии в последние полвека остро возникла проблема чрезмерной загруженности транспортных потоков как в городах, так и на автомагистралях. Для решения этой проблемы по всему миру общественностью принимается множество мер:
\begin{itemize}
\item Депопуляризации личного транспорта. 23 сентября каждого года по всему миру проводится всемирный ''день без автомобиля'' (\url{https://opennov.ru/news/society/2019-09-22/20939})
\item Создание программ развития общественного транспорта (\url{https://www.mintrans.ru/images/content/gos-programma-rasv-tran-sist.pdf}). Известно, что каждый автобус или вагон метро средней заполненности приблизительно в 20 раз более эффективен с точки зрения загрузки транспортного потока, чем аналогичное по вместительности пассажиров количество личных автомобилей. Развивая инфраструктуру общественного транспорта, возможно значительно разгрузить дорожные пути.
\item Оптимизация системы дорожного движения. Для обеспечения эффективного дорожного движения необходимо отрегулировать систему управления транспортными потоками таким образом, чтобы создать максимально однородное по скорости и координатам распределение автомашин. В отличие от предыдущих двух, эта мера действует как для городских, так и для магистральных транспортных путей.
\end{itemize}

Целью данной работы является исследование существующих вероятностных моделей дорожного движения на автомагистрали. Имея математическую модель, возможно воссоздать более точную картину проблем, связанных с дорожным движением на автомагистралях, и путей их решения.
\section{Описание вероятностной модели}

Мы будем рассматривать одномерный поток автомобилей на автомагистрали. Введем две  случайных величины:

\begin{itemize}
\item $\eta(t), t > 0$ - число машин, пересекших некоторую случайную линию с координатой \textit{x} за промежуток времени (0, t). При этом $\eta(0) = 0$.
\item $\nu(x), x > 0$ - число машин, располагающихся на участке дороги (0,x) в момент времени t. При этом $\nu(0) = 0$.
\end{itemize}

Предположим, что на дороге существует два типа автомобилей: ''быстрые'' и ''медленные''. Наблюдения показывают, что движение на автомагистрали при плохих погодных условиях происходит по следующему принципу: из-за сложности совершения обгона в таких условиях образуются колонны (пачки) ''быстрых'' автомобилей, возглавляемые "медленным" автомобилем. Колонна пополняется, когда ее догоняет очередной "быстрый" автомобиль, и уменьшается, когда происходит обгон головного "медленного" автомобиля. Пачки автомобилей при этом считаются независимыми.

\subsection{Идеальные условия}
При хороших условиях (прямой участок дороги с хорошей видимостью и сухом дорожном покрытии) движение автомобилей по автомагистрали является совокупностью независимых движений каждого автомобиля. С точки зрения дорожного движения это связано с тем, что маневр обгона не представляет сложности, и задержка при опережении медленного автомобиля быстрым минимальна. В таком случае распределение являетcя пуассоновским.
\textit{**Место для доказательства; ссылка 136 в диссертации**}

\subsection{Плохие погодные условия}
При плохих погодных условиях преположение о независимости движений автомобилей, движущихся с разной скоростью, не выполняется. В данном случае необходима более сложная модель.

Введем следующие обозначения:
\begin{itemize}

\item $\eta_0(t)$ - случайное число быстрых автомобилей, поступающих в колонну. Исходя из предыдущего раздела,
$P(\{\eta_0(t) = k\}) = \frac{\lambda_0^k}{k!}e^{- \lambda_0}$,
где $\lambda_0$ - параметр распределения Пуассона, интерпретируемый как интенсивность поступления ''быстрых'' машин в пачку.

\item $\varkappa_0(t)$ - случайная величина, обозначающая количество автомобилей всех типов в пачке, принимающее значения {1 ... N}.

\item $\xi_0(t, \Delta t)$ - случайная величина, обозначающая число быстрых автомобилей, совершающих обгон в промежуток времени $(t, t + \Delta t)$
\end{itemize}

Можно получить следующие равенства:

\begin{enumerate} 
\item Так как движение быстрых машин на автомагистрали можно считать равномерным, они поступают в пачку по линейному закону. При малых $\Delta t$ можно записать:

$P(\eta_0(t+ \Delta t) - \eta_0(t) = 0) = 1 - \lambda_0 \Delta t + o(\Delta t)$

$P(\eta_0(t+ \Delta t) - \eta_0(t) = 1) = \lambda_0 \Delta t - o(\Delta t)$

\item Предположим, что движение достаточно разреженное, чтобы процессы внутри каждой пачки не оказывали влияния на количество поступающих в нее машины. Тогда $\forall \ t_2 > t_1 \ P(\{\eta_0(t_2) - \eta_0(t_1) = k\})$ зависит только от \textit{k} и разности $t_2 - t_1$, и следовательно $P(\{\eta_0(t_2) - \eta_0(t_1) = k\}) = P({\eta_0(t_2 - t_1) = k}) \ \forall \ k$.

% \end{enumerate} 
\suspend{enumerate}
Пользуясь (1) и условием $\sum_{k = 0}^{\infty}P(\{\eta_0(t+\Delta t) - \eta_0(t) = k\}) = 1$, можно получить, что $P(\{\eta_0(t+\Delta t) - \eta_0(t) \geq 2 \}) = o(\Delta t)$ - вероятность того, что при малом $\Delta t$ пачку догонит более одной машины, мала. Это согласуется с правилами дорожного движения. Такое соображение позволяет не рассматривать события, связанные с изменением величин $\eta_0, \xi, \varkappa$ более, чем на 1, так как их вероятность мала.

Рассмотрим условные вероятности событий:
\resume{enumerate}
% \begin{enumerate}
\item $P(\{\xi_0(t, \Delta t) = 0\} | \{ \varkappa_0(t, \Delta t) = 1, \eta_0(t, \Delta t) = 0\}) = 1$

Физический смысл этого равенства состоит в том, что автомобиль не может совершить обгон, будучи единственным в пачке.

\item $P(\{\xi_0(t, \Delta t) = 0\} | \{ \varkappa_0(t, \Delta t) = 1, \eta_0(t, \Delta t) = 1\}) = 1 - O(\Delta t)$

Это равенство означает, что машина, только что присоединившаяся к колонне, не будет совершать обгон с линейно убывающей со временем вероятностью.

\item 

\end{enumerate} 

\subsection{}
\section{Экспериментальная проверка модели}

\section{Заключение}

\section{Список литературы}
\end{document}